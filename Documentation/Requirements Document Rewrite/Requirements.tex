\documentclass[letterpaper]{article}
    \usepackage{amsmath}
    \usepackage{tikz}
    \usepackage{epigraph}
    \usepackage{lipsum}
    \setcounter{tocdepth}{1}
    \renewcommand\epigraphflush{flushright}
    \renewcommand\epigraphsize{\normalsize}
    \setlength\epigraphwidth{0.7\textwidth}
    
    \definecolor{titlepagecolor}{cmyk}{1,.60,0,.40}
    
    \DeclareFixedFont{\titlefont}{T1}{ppl}{b}{it}{0.5in}
    
    \makeatletter                       
    \def\printauthor{%                  
        {\large \@author}}              
    \makeatother
    \author{
        Brittany Lacy \\
        Drake Lambert \\  
        Jeremy LeJeune \\
        Jenna Meadors \\
        Sam Miller \\
       }
    
    % The following code is borrowed from:  https://tex.stackexchange.com/a/86310/10898
    
    \newcommand\titlepagedecoration{%
    \begin{tikzpicture}[remember picture,overlay,shorten >= -10pt]
    
    \coordinate (aux1) at ([yshift=-15pt]current page.north east);
    \coordinate (aux2) at ([yshift=-410pt]current page.north east);
    \coordinate (aux3) at ([xshift=-4.5cm]current page.north east);
    \coordinate (aux4) at ([yshift=-150pt]current page.north east);
    
    \begin{scope}[titlepagecolor!40,line width=12pt,rounded corners=12pt]
    \draw
      (aux1) -- coordinate (a)
      ++(225:5) --
      ++(-45:5.1) coordinate (b);
    \draw[shorten <= -10pt]
      (aux3) --
      (a) --
      (aux1);
    \draw[opacity=0.6,titlepagecolor,shorten <= -10pt]
      (b) --
      ++(225:2.2) --
      ++(-45:2.2);
    \end{scope}
    \draw[titlepagecolor,line width=8pt,rounded corners=8pt,shorten <= -10pt]
      (aux4) --
      ++(225:0.8) --
      ++(-45:0.8);
    \begin{scope}[titlepagecolor!70,line width=6pt,rounded corners=8pt]
    \draw[shorten <= -10pt]
      (aux2) --
      ++(225:3) coordinate[pos=0.45] (c) --
      ++(-45:3.1);
    \draw
      (aux2) --
      (c) --
      ++(135:2.5) --
      ++(45:2.5) --
      ++(-45:2.5) coordinate[pos=0.3] (d);   
    \draw 
      (d) -- +(45:1);
    \end{scope}
    \end{tikzpicture}%
    }
    
\begin{document}
\begin{titlepage}

	\noindent
	\titlefont The Golden Ticket\par
	\epigraph{Software Requirements Document}%
	{\textit{}\\ \textsc{}}
	\null\vfill
	\vspace*{1cm}
	\noindent
	\hfill
	\begin{minipage}{0.35\linewidth}
		\begin{flushright}
			\printauthor
		\end{flushright}
	\end{minipage}
	%
	\begin{minipage}{0.02\linewidth}
		\rule{1pt}{125pt}
	\end{minipage}
	\titlepagedecoration
\end{titlepage}
\tableofcontents
\pagebreak
% This defines the expected readership of the document and describes its version history, including the rationale for the creation of a new version and a summary of the changes made in each version.
\section{Preface}
This document is a complete rewrite of the previously submitted version in order to better fit the format required of the requirements document. This document will contain all of the information of the previous document, and will additionally contain the sections as suggested in the textbook.

In making this document we intend to layout the master plan for a ticketing system using the dot net core.
\pagebreak

% This describes the need for the system. It should briefly describe the system's functions and explain how it will work with other systems. It should also describe how the system fits into the overall business or strategic objectives of the organization commissioning the software.
\section{Introduction}
Golden Ticket is designed to be a light weight adaptable ticketing system. There will be two different kinds of users for this system, the technicians that work the tickets and the administrators that oversee them. The technicians will have access to their assigned tickets, have properties that denote their experience and years with the company. 

\pagebreak

% This defines the technical terms used in the document. You should not make assumptions about the experiences or expertise of the reader.
\section{Glossary}
\lipsum

% Here, you describe the services provided for the user. The nonfunctional system requirements should also be described in this section. This description may use natural language, diagrams, or other notations that are understandable to customers. Product and process standards that must be followed should be specified.
\section{User Requirements definition}
\lipsum

% This chapter presents a high-level overview of the anticipated system architecture, showing the distribution of functions across system modules. Architectural components that are reused should be highlighted.
\section{System Architecture}
\lipsum

% This describes the functional and nonfunctional requirements in more detail. If necessary, further detail may also be added to the nonfunctional requirements. Interfaces to other systems may be defined.
\section{System Requirements Specification}
\lipsum
\pagebreak

% This chapter includes graphical system models showing the relationships between the system components and the system and its environment. Examples of possible models are object models, data-flow models, or semantic data models.
\section{System Models}
\lipsum

% This describes the fundamental assumptions on which the system is based, and any anticipated changes due to hardware evolution, changing user needs, and so on. This section is useful for system designers as it may help them avoid design decisions that would constrain likely future changes to the system.
\section{System Evolution}
\lipsum

% These provide detailed, specific information that is related to the application being developed—for example, hardware and database descriptions. Hardware requirements define the minimal and optimal configurations for the system. Database requirements define the logical organization of the data used by the system and the relationships between data.
\section{Appendices}
\lipsum

% Several indexes to the document may be included. As well as a normal alphabetic index, there may be an index of diagrams, an index of functions, and so on.
\section{Index}
\lipsum



\end{document}